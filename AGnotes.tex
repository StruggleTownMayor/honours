%% LyX 2.0.2 created this file.  For more info, see http://www.lyx.org/.
%% Do not edit unless you really know what you are doing.
\documentclass[english]{article}
\usepackage[T1]{fontenc}
\usepackage[latin9]{inputenc}
\usepackage{geometry}
\geometry{verbose,tmargin=2cm,lmargin=2cm,rmargin=2cm}
\usepackage{amsbsy}
\usepackage{amssymb}
\usepackage{babel}
\begin{document}
\global\long\def\A{\mathbb{A}}


\global\long\def\P{\mathbb{P}}


\global\long\def\R{\mathbb{R}}


\global\long\def\C{\mathbb{C}}



\section{Commutative Algebra}
\begin{itemize}
\item An ideal $I$ is prime if $fg\in I$ means $f\in I$ or $g\in I$
\item An element $p$ is prime if $p|fg$ means $p|f$ or $p|g$
\item An element $p$ is irreducible if it cannot be factored into two non-invertible
elements
\item In a UFD, irreducible and prime are equivalent
\item A polynomial ring over a UFD is a UFD
\item A ring is Noetherian if every ascending chain of ideals is stationary,
or equivalently every ideal is finitely generated
\item A polynomial ring over a Noetherian ring is Noetherian
\item prime ideals are radical
\item quotient of a prime ideal is an integral domain
\end{itemize}

\section{Topology}
\begin{itemize}
\item Zariski topology is where closed sets are zero loci
\item An irreducible set is not a proper union of closed subsets
\item continuous images of irreducible sets are irreducible
\item products of irreducible sets are irreducible
\item A Noetherian space is where every decreasing chain of closed subsets
is stationary
\item In a Noetherian space decomposition into irreducible components is
unique
\item The dimension of an irreducible space is the length of the longest
chain of nonempty irreducible closed subsets minus 1; the dimension
of any space is the maximum dimension of an irreducible component
\item product topology is weaker than zariski topology on product
\end{itemize}

\section{Abstract Nonsense}
\begin{itemize}
\item A presheaf of rings is an assignment of rings to open sets, as well
as a restriction homomorphism, in such a way that restriction is transitive
and reflexive, and the empty set is assigned to the trivial ring.
\item A sheaf is a presheaf with the property that a ring element on the
global ring can be defined by its restriction to the sets of an open
cover
\item The stalk of a presheaf is the set of all ring elements of all neighborhoods
of $P$, identifying elements if they agree on some neighborhood of
$P$.
\item germs are the elements of a stalk
\item the stalk of $O_{X}$ at $P$ is $O_{X,P}$
\item If $f:X\to Y$ is a function, the pull-back $f^{*}\phi$ of a regular
function $\phi$ on $Y$ is $\phi\circ f$
\item A function is a morphism if it pulls back regular functions to regular
functions
\item A morphism of sheaves $F,G$ is collection of morphisms $f_{U}:F\left(U\right)\to G\left(U\right)$
that commute with restriction maps
\item a rational map is a morphism from an open subset, under the equivalence
of agreeing on an open set. A rational map is dominant if a representative
has a dense image; it is birational if it has a rational inverse.
\end{itemize}

\section{Affine Varieties}
\begin{itemize}
\item Affine variety corresponds to prime ideal
\item The coordinate ring $A\left(X\right)=k\left[x_{i}\right]_{i}/I\left(X\right)$
is the ``polynomials'' on $X$
\item The rational functions $K\left(X\right)$  are the fraction field
of $A\left(X\right)$
\item The regular functions at $P$ are those rational functions that can
be evaluated at $P$; alternatively they are the functions that have
a representation as a quotient of polynomials in $k\left[x_{i}\right]_{i}$
on some neighborhood of $P$.
\item The regular functions on an open set $U$ are the functions that are
regular at each $P\in U$.
\item $O\left(X_{f}\right)=A\left(X\right)_{f}:=\left\{ g/f^{r}\right\} \subseteq K\left(X\right)$
\item A product of affine varieties in $\A^{n}$ and $\A^{m}$ is an affine
variety in $\A^{n+m}$ (not product topology)
\item An abstract affine variety is an irreducible space and a sheaf of
$k$-valued functions that is isomorphic to a concrete affine variety
\item Distinguished open subsets are abstract affine varieties
\item Not all open subsets of varieties are varieties, consider $\C^{2}\backslash\left\{ \left(0,0\right)\right\} $
\item morphisms can be checked on open sets, germs or global sets
\item morphisms of affine varieties $f$ correspond to $k$-algebra homomorphisms
$f^{*}$
\end{itemize}

\section{Varieties}
\begin{itemize}
\item A prevariety is an irreducible set with a sheaf of functions that
has a finite cover of affine varieties
\item Can create a prevariety by gluing two prevarieties along a common
open subset: Let $f$ be the isomorphism between open subsets $U_{1},U_{2}$
of $X_{1},X_{2}$ and $i_{1},i_{2}$ be the inclusions into $X$.
The topology is the quotient topology and the sheaf of functions is
pairs $\left(\phi_{1},\phi_{2}\right)\in O_{X_{1}}\left(i_{1}^{-1}\left(U\right)\right)\times O_{X_{2}}\left(i_{2}^{-1}\left(U\right)\right)$
that agree on overlaps
\item Same thing works with finite collection of prevarieties, with each
pair glued on an open subset, provided isomorphisms are consistent.
\item Let $\left\{ V_{i}\right\} $ be an affine cover of $Y$ and $\left\{ U_{i}\right\} $
be an open cover of $X$ with $f\left(U_{i}\right)\subseteq V_{i}$
and $f$ a morphism when restricted to each $U_{i}$. Then $f$ is
a morphism.
\item A variety is a prevariety $X$ so that for any prevariety $Y$ and
pair of morphisms $Y\to X$, the set where they agree is closed; equivalently,
the diagonal is closed.
\item an open or closed subprevariety of a variety is a variety.
\item A variety is complete if $\pi:X\times Y\to Y$ is closed for every
variety $Y$
\item If $X$ is complete then any morphism $X\to Y$ ($Y$ variety) is
closed.
\item regular functions on complete varieties are constant
\item Let $X,Y$ be affine varieties with $Y$ affine. Morphisms $X\to Y$
are in correspondence with homomorphisms $A\left(Y\right)\to O\left(X\right)$.
If the morphism $f$ is $\left(f_{i}\right)_{i=1}^{\dim X}$, $f_{i}$
regular, this corresponds to $f^{*}:\bar{x}_{i}\mapsto f_{i}$.
\item The rational functions on a general variety are equivalence classes
of regular functions that agree on open sets
\item rational maps from $X$ to $Y$ are equivalent to homomorphisms between
$K\left(Y\right)$ and $K\left(X\right)$.
\end{itemize}

\section{Projective Space}
\begin{itemize}
\item think of $\P^{n}$ as $\A^{n}$ compactified with a point at infinity
for every direction
\item A projectivity on $\P^{n}$ is an element of $\mbox{GL}_{n+1}\left(k\right)/k^{*}$
\item A conic is a symmetric bilinear form on $k^{3}$, which can be represented
as $\varepsilon_{1}X^{2}+\varepsilon Y^{2}+\varepsilon Z^{2}$ after
a projectivity
\item conics in $\P_{\R}^{2}$ are

\begin{itemize}
\item nondegenerate $X^{2}+Y^{2}-Z^{2}$
\item empty $X^{2}+Y^{2}+Z^{2}$
\item one point $X^{2}+Y^{2}$
\item two lines $X^{2}-Y^{2}$
\item line $X^{2}$
\item everything $0$
\end{itemize}
\item nondegenerate conics are equivalently $XY=Z^{2}$, isomorphic to $\P^{1}$
with the isomorphism $\left(U:V\right)\mapsto\left(U^{2},UV,V^{2}\right)$
which can be interpreted as projection
\item conics in $\P_{\bar{k}}^{2}$

\begin{itemize}
\item nondegenerate $X^{2}+Y^{2}-Z^{2}$
\item two lines $X^{2}-Y^{2}$
\item line $X^{2}$
\item everything $0$
\end{itemize}
\item A degree-$d$ homogeneous form $F$ on $\P^{n}$ corresponds to a
(maximum) degree-$d$ polynomial $f$ in $\A^{n}$. If $n=1$, the
multiplicity of a zero in $F$ is the multiplicity of the corresponding
zero in $f$, or $d-\deg f$ for the point at infinity.
\item Bezout's theorem: for an algebraically closed field the number of
intersections of projective curves is the product of their degrees,
provided they share no irreducible components and multiplicities are
counted appropriately
\item Easy cases: line or nondegenerate conic vs a nonincluding curve in
$\P^{2}$, inequality to compensate for multiplicities
\item 5 points in general position define a unique conic
\end{itemize}

\section{Projective Varieties}
\begin{itemize}
\item Homogeneous ideals are generated by homogeneous polynomials or equivalently
contain each homogeneous part of each member, or equivalently are
fixed by the action of $k^{*}$.
\item A projective algebraic set $X$ in $\P^{n}$ corresponds to a cone
$C\left(X\right)$ in $\A^{n+1}$
\item The zero set of a homogeneous ideal in $\A^{n+1}$ is the cone of
its zero set in $\P^{n}$ (provided neither are empty)
\item the ideal generated by $X\subseteq\P^{n}$ is the ideal generated
by $C\left(X\right)\in\A^{n+1}$.
\item $\dim X+1=\dim C\left(X\right)$ (provided nonempty)
\item Nullstellensatz still works provided $Z\left(I\right)$ is nonempty;
$Z\left(I\right)$ can only be empty if $I=\left\langle 1\right\rangle $
or $\sqrt{I}=\left\langle x_{0},\dots,x_{n}\right\rangle $.
\item Homogeneous coordinate ring is $S\left(X\right)=A\left(C\left(X\right)\right)$;
\textbf{not} polynomial functions
\item rational functions are $f/g$, where $f,g\in S\left(X\right)^{\left(d\right)}$
have common degree $d$
\item homogeneous functions of the same degree on homogeneous coordinates
give a morphism, provided they never all vanish
\item projective varieties are varieties
\item The Segre embedding is $\P^{n}\times\P^{m}\to\P^{\left(n+1\right)\left(m+1\right)-1}:\left(\left(x_{i}\right),\left(y_{i}\right)\right)\mapsto\left(x_{i}y_{j}\right)$.
It is the zero locus of $z_{i,j}z_{i',j'}-z_{i,j'}z_{i',j}$.
\item projective varieties are complete
\item A nontrivial projective variety intersects with the zero locus of
any homogeneous polynomial
\item The Veronese embedding is $\P^{n}\to\P^{{n+d \choose n}-1}:\left(x=\left(x_{i}\right)\right)\mapsto\left(x^{I}\right)_{I}$
(monomials of degree $d$). It is the zero locus of $z_{I}z_{J}-z_{K}z_{L}$,
with $I+J=K+L$
\item The degree of a projective variety is the maximal finite number of
intersection points with a linear subvariety that fills the remaining
dimensions. The degree of $Z\left(F\right)$ is $\deg\left(F\right)$
\item Let $B=\left\{ \left(x;\ell\right)\in\A^{n}\times\P^{n-1}:\ell\mbox{ passes through }x\right\} $,
and let $\pi$ be the projection $\left(x,\ell\right)\to x$. Then
the closure of $\pi^{-1}\left(X\backslash\left\{ p\right\} \right)$
is the blowup of $X$ at $p$.
\item The Grassmannian $G\left(k,n\right)$ is the set of $k-1$ dimensional
linear subvarieties of $\P^{n-1}$, and can be embedded as a variety
in $\P^{{n \choose k}-1}$ with the Pl\"ucker embedding $\mbox{span}\left(v_{i}\right)_{i}\mapsto\bigwedge_{i}v_{i}$
\item There are 27 projective lines on any smooth cubic surface
\end{itemize}

\section{Dimension}
\begin{itemize}
\item projective morphisms cannot map surjectively onto varieties with higher
dimension
\item projection to a point doesn't decrease dimension
\item Adding a single polynomial to a zero locus decreases its dimension
by 1.
\item open subsets of a variety have the same dimension
\item all the components of the zero locus in $\P^{n}$ or $\A^{n}$ of
a single polynomial have dimension $n-1$; intersecting such a zero
locus with a variety decreases its dimension by 1.
\item If $f:X\to Y$ is a morphism with $\dim\left(f^{-1}p\right)\equiv n$
then $\dim X=\dim Y+n$
\item dimension of $X$ is transcendence degree of $K\left(X\right)$ (maximal
number of algebraically independent elements over $k$)
\item dimension of affine $X$ is the Krull dimension (maximal size of a
chain of prime ideals) of $A\left(X\right)$
\end{itemize}

\section{Smoothness}
\begin{itemize}
\item A line $\ell=Z\left(g\right)$ is tangent to $X$ with ideal $\left\langle f_{i}\right\rangle _{i}$
at $p$ if the intersection of $f$ and $g$ has multiplicity at least
2
\item tangent space is union of all tangent lines
\item equivalently, tangent space is zero locus of linear parts of generators
(can compute with taylor expansion)
\item tangent space is isomorphic to dual of $\mathfrak{m}_{p}/\mathfrak{m}_{p}^{2}$
(zariski tangent), where $\mathfrak{m}_{p}$ is the maximal ideal
$\left\{ f\in O_{X,p}\left(X\right):f\left(p\right)=0\right\} $ of
$O_{X,p}$
\item tangent space of $X$ has at least local dimension of $X$. Smooth
if dimension is same, singular otherwise
\item compute dimension of tangent space with rank of jacobian
\item The set of singularities of a closed set is a closed set
\item Hironaka's theorem: for any projective variety $V$ there is a smooth
desingularization $X$ which is birational to $X$, obtained by blowups\end{itemize}

\end{document}
