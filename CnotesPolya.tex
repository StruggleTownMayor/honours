%% LyX 2.0.3 created this file.  For more info, see http://www.lyx.org/.
%% Do not edit unless you really know what you are doing.
\documentclass[oneside,english]{amsart}
\usepackage[T1]{fontenc}
\usepackage[latin9]{inputenc}
\setlength{\parskip}{\bigskipamount}
\setlength{\parindent}{0pt}
\usepackage{amsthm}
\usepackage{amssymb}

\makeatletter
%%%%%%%%%%%%%%%%%%%%%%%%%%%%%% Textclass specific LaTeX commands.
\numberwithin{equation}{section}
\numberwithin{figure}{section}

\makeatother

\usepackage{babel}
\begin{document}

\section{P\'olya Counting}

Texts:
\begin{itemize}
\item \textbf{van Lint and Wilson}
\item Keller and Trotter
\item Brualdi
\end{itemize}
P\'olya counting is a method for counting colourings in the presence
of symmetry. It can also be used to count unlabelled objects.

\textbf{Example:} Consider a square. We will colour the vertices of
the square black or white. If the square is fixed in space there are
16 possibilities. But, if rotations and reflections are allowed, then
some of these become equivalent. We now want to count/list colourings
up to symmetry. There are only 6.

\textbf{Group theory Revision:} definition of a group, left cosets,
Lagrange's theorem. group actions: orbit $G\cdot s$, stabilizer $\mbox{stab}_{G}s=\left\{ g:g\cdot s=s\right\} $,
$\mbox{fix}_{S}g=\left\{ s:g\cdot s=s\right\} $, set of orbits $G\backslash S$

\textbf{Example: }The group of symmetries of the square acts on the
set of vertex labels $\left\{ 1,2,3,4\right\} $ (not yet on colourings)
as follows: rotation corresponds to $\left(1234\right)$ or $\left(13\right)\left(24\right)$
or $\left(1432\right)$, reflections correspond to blah (there are
4). These are the only symmetries of the square. There is a single
orbit $G\backslash S$. $\mbox{stab}_{G}1=\left\{ \left(\right),\left(24\right)\right\} $,
$\mbox{fix}_{S}\left(24\right)=\left\{ 1,3\right\} $.

The group of symmetries of the square is the dihedral group with order
8. In general, the group of symmetries of a regular $n$-gon is the
dihedral group $D_{2n}$ of order $2n$, generated by a rotation by
$2\pi/n$ and a reflection.

Now suppose $G$ is a group which acts on a finite set $A$, and let
$B$ be a finite set of colours. Then $G$ acts on the set $B^{A}$
of functions from $A$ to $B$ as follows: given an element $g\in G$
and a $f\in B^{A}$, define $g\cdot f\in B^{A}$ by $\left(g\cdot f\right)\left(a\right)=f\left(g^{-1}\cdot a\right)$
for all $a\in A$. We will check that this defines a $G$-action.

\emph{Proof:} Clearly $1\cdot f=f$ for all $f\in B^{A}$, where 1
denotes the identity of $G$. Next, let $g,h\in G$ and $f\in B^{A}$.
Then 
\begin{align*}
\left(gh\cdot f\right)\left(a\right) & =f\left(\left(gh\right)^{-1}\cdot a\right)\\
 & =f\left(\left(h^{-1}g^{-1}\right)\cdot a\right)\\
 & =f\left(h^{-1}\cdot\left(g^{-1}\cdot a\right)\right)\\
 & =f\left(h^{-1}\cdot\left(g^{-1}\cdot a\right)\right)
\end{align*}
Also, 
\begin{align*}
\left(g\cdot\left(h\cdot f\right)\right)\left(a\right) & =\left(h\cdot f\right)\left(g^{-1}\cdot a\right)\\
 & =f\left(h^{-1}\cdot g^{-1}\cdot a\right)
\end{align*}
so $\left(gh\right)\cdot f=g\cdot\left(h\cdot f\right)$ for all $g,h\in G$,
$f\in B^{A}$, as required.\qed

We wish to count orbits of the action of $G$ on $B^{A}$, which is
the same as counting inequivalent $B$-colourings of $A$ (under the
action of $G$).

\textbf{Typical question:} How many inequivalent colourings of faces
of a cube are there, with $k$ colours, up to rotation?


\subsection{Burnside's Lemma}

First we prove

\textbf{Proposition: }Let $G$ be a finite group which acts on the
finite set $S$. Then, for all $s\in S$, 
\[
\sum_{\hat{s}\in G\cdot s}\left|\mbox{stab}_{G}\left(\hat{s}\right)\right|=\left|G\right|.
\]
\emph{Proof (combinatorial):}\textbf{ }Write $\mbox{stab}_{G}\left(s\right)=\left\{ g_{1},g_{2},\dots,g_{k}\right\} $
($g_{i}$ distinct) for some $k\in\mathbb{Z}^{+}$. For any $s'\in G\cdot s$,
define $T\left(s,s'\right)=\left\{ g\in G:g\cdot s=s'\right\} $.
Note, $T\left(s,s\right)=\left\{ g\in G:g\cdot s=s\right\} =\mbox{stab}_{G}\left(s\right)$.
Fix $s'\in G\cdot s$ and choose $g\in T\left(s,s'\right)$. Then
$gg_{i}\in T\left(s,s'\right)$ for $i=1,\dots,k$. If $gg_{i}=gg_{j}$
then $g_{i}=g_{j}$, so $i=j$. Hence, the function $\mbox{stab}_{G}\left(s\right)\to T\left(s,s'\right):g_{i}\mapsto gg_{i}$
is injective. Furthermore, if $g'\in T\left(s,s'\right)$ then $g^{-1}g'\in\mbox{stab}_{G}\left(s\right)$,
so $g^{-1}g'=g_{j}$ for some $j\in\left\{ 1,\dots,k\right\} $. Hence
$g'=gg_{j}$, which shows the above map is onto. Hence the map is
a bijection and so 
\begin{equation}
\left|\mbox{stab}_{G}\left(s\right)\right|=\left|T\left(s,s'\right)\right|\label{eq:I}
\end{equation}
{[}Note, $s'\in G\cdot s$ was arbitrary.{]}

Also, for all $s'\in G\cdot s$ we have 
\begin{equation}
T\left(s',s\right)=\left\{ g^{-1}:g\in T\left(s,s'\right)\right\} .\label{eq:II}
\end{equation}
Hence $\left|\mbox{stab}_{G}\left(s'\right)\right|=\left|T\left(s',s\right)\right|=\left|T\left(s',s\right)\right|=\left|\mbox{stab}_{G}\left(s\right)\right|$
by (\ref{eq:I}) and (\ref{eq:II}).

Therefore
\[
\sum_{\hat{s}\in G\cdot s}\left|\mbox{stab}_{G}\left(\hat{s}\right)\right|=\sum_{\hat{s}\in G\cdot s}\left|T\left(s,\hat{s}\right)\right|
\]
by (\ref{eq:I}) and each element of $G$ appears in precisely one
set $T\left(s,\hat{s}\right)$. Therefore the right hand side of the
above expression equals $\left|G\right|$, as required.\qed

\textbf{Lemma (Burnside's Lemma):} let $G$ be a finite group which
acts on the finite set $S$, and let $N=\left|G\backslash S\right|$
be the number of $G$-orbits of $S$. Then $N=\frac{1}{\left|G\right|}\sum_{g\in G}\left|\mbox{fix}_{S}\left(g\right)\right|$,
the average number of fixed points.

\emph{Proof:}\textbf{ }Let $\mathcal{A}=\left\{ \left(g,s\right):g\in G,s\in S\mbox{ and }g\cdot s=s\right\} $.
Apply double counting. Counting by $g$ first gives $\left|\mathcal{A}\right|=\sum_{g\in G}\left|\mbox{fix}_{S}\left(g\right)\right|$,
and counting by $s$ gives 
\[
\left|\mathcal{A}\right|=\sum_{s\in S}\left|\mbox{stab}_{G}\left(s\right)\right|=\sum_{U\in G\backslash S}\sum_{\hat{s}\in U}\left|\mbox{stab}_{G}\left(\hat{s}\right)\right|=\sum_{U\in G\backslash S}\left|G\right|=N\left|G\right|.
\]
\qed

Specialize to our setting:

\textbf{Theorem: }Let $A,B$ be finite sets and let $G$ be a finite
group which acts on $A$. For $i\in\mathbb{Z}^{+}$, let $c_{i}\left(G\right)$
be the number of elements of $G$ which have exacty $i$ cycles in
their disjoint cycle decomposition, considered as permutations of
$A$. Then, the number of orbits of $G$ on $B^{A}$ is 
\[
\frac{1}{\left|G\right|}\sum_{i=1}^{\infty}c_{i}\left(G\right)\left|B\right|^{i}=\mbox{\# of inequivalent \ensuremath{B}-colourings unde the action of \ensuremath{G}.}
\]
\emph{Proof:} By Burnside's Lemma, the answer is 
\[
\frac{1}{\left|G\right|}\sum_{g\in G}\Psi\left(g\right),
\]
where $\Psi\left(g\right)$ is the number of colourings $f\in B^{A}$
which are fixed by $g$. Suppose that $g$ has $i$ cycles in its
cycle decomposition (as a permutation of $A$). Then $f\in B^{A}$
is fixed by $g$ iff $f$ is a constant on each of these $i$ cycles.
Hence there are exactly $\left|B\right|^{i}$ colourings $f\in B^{A}$
fixed by $g$, and the result follows.\qed

Recall, symmetries of the square, acting on vertex labels. $\left|B\right|=2$.
$\left(\right)$ fixes $2^{4}$ colourings (all of them). $\left(1234\right)$
fixes $2^{1}$colourings, etc

By the theorem, the number of inequivalent 2-colourings of the vertices
of a square is $\frac{1}{8}\left(1\times2^{4}+2\times2^{3}+3\times2^{2}+2\times2^{1}\right)=6$
\end{document}
